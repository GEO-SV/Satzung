\documentclass[10pt,a4paper,oneside,parskip=half]{scrartcl}
\usepackage[utf8]{inputenc}
\usepackage[ngerman]{babel}
\usepackage[T1]{fontenc}
\usepackage[margin=2.5cm]{geometry}
\usepackage{lmodern}
\usepackage{graphicx}
\usepackage{tabularx}
\usepackage[juratotoc]{scrjura}
\makeatletter
\providecommand*{\toclevel@cpar}{0}
\makeatother
\usepackage{lastpage}
\usepackage{scrpage2}
\usepackage{hyperref}

\title{Satzung}
\date{\today}
\author{Schülervertretung Gymnasium Eversten Oldenburg}

\pagestyle{scrheadings}
\renewcommand*{\titlepagestyle}{scrheadings}

\begin{document}
\cfoot{Seite \thepage\ von \pageref{LastPage}}
\maketitle

\begin{contract}

\Clause{title=Aufgaben und Ziele der Schülervertretung}
Die Schülermitwirkung ist grundsätzliche Aufgabe und recht \emph{aller} Schüler*innen des Gymnasium Eversten Oldenburg.

Die Organe der Schülervertretung sind die Vollversammlung, welche sich aus \emph{allen Klassen-, Kurs-, Jahrgangs- und Schülersprecher*innen sowie den freiwilligen Schülervertretungs-Mitglieder*innen} zusammensetzt -- sowie das Schülervertretungs-Team, bestehend aus freiwilligen Mitglieder*innen, Jahrgangssprecher*innen und Schülersprecher*innen.

Die primäre Aufgabe der Schülervertreter*innen ist es, die Interessen der Schüler*innen des Gymnasium Eversten Oldenburg in der Gesamtkonferenz, in Fachkonferenzen, im Schulvorstand und gegenüber dem Schulelternrat sowie der Schulleitung zu vertreten. Hierzu gehören beispielsweise die Wahrnehmung von Schülerinteressen sowie die Mithilfe bei der Lösung von Konflikten und die Durchführung gemeinsamer Veranstaltungen.

Zu den Rechten und Pflichten der Schülervertretung nach §~80~NSchG gehören insbesondere:
\begin{itemize}
\item In \emph{allen} sie betreffenden Angelegenheiten durch die Schule informiert zu werden;
\item Vor grundsätzlichen, vor allem die Organisation der Schule oder die Leistungsbewertung betreffenden, Entscheidungen \emph{von der Schulleitung oder der zuständigen Konferenz gehört zu werden};
\item \emph{Wünsche und Anregungen der Schülerschaft} an die Lehrerschaft, die Schulleitung und den Schulelternrat zu übermitteln;
\item Auf Antrag eines betroffenen Schülers / einer betroffenen Schülerin ihre \emph{Hilfe und Vermittlung} einzusetzen, wenn diese(r) glaubt, ihm sei Unrecht geschehen;
\item \emph{Beschwerden allgemeiner Art} bei der Lehrerschaft, der Schulleitung und in der Gesamtkonferenz vorzubringen;
\item jährlich \emph{bis zu vier zweistündige Schülerversammlungen}, während der Unterrichtszeit, abzuhalten.
\end{itemize}

\Clause{title=Grundsätze}
Schüler*innen dürfen wegen ihrer Tätigkeit als Schülervertreter \emph{weder bevorzugt noch benachteiligt} werden. Auf Antrag der Schüler*innen ist die Tätigkeit als Vertreter \emph{im Zeugnis oder in anderer geeigneter Form} ohne Wertung zu bescheinigen.
	
Schülervertreter sind \emph{ehrenamtlich} tätig und bei der Erfüllung ihrer Aufträge und Weisungen \emph{nicht gebunden}. Sie sind in ihren Entscheidungen und Handlungen nur gegenüber der Schülerschaft verantwortlich.

Jegliche Einnahmen der Schülervertretung sind im Sinne der Schülerschaft einzusetzen

Sämtliche Wahlen der Schülervertretung sind allgemein, frei, gleich und geheim. Alle Ämter (mit Ausnahme des Schülersprechers nach §~5~Abs.~4) werden \emph{für die Dauer eines Schuljahres} oder bis zum Rücktritt der amtierenden Person vergeben.

Damit die Schülervertretung beschlussfähig ist, müssen zur entsprechenden Sitzung \emph{mindestens $\frac{2}{3}$ der Mitglieder} anwesend sein.

Die Schlüssel für den Schülervertretungs-Raum werden an die beiden Schülersprecher*innen, den/die Unterstufensprecher*in sowie nach Absprache an andere Schülervertretungs-Mitglieder*innen vergeben.

Der Schülervertretungs-Raum ist grundsätzlich verantwortungsvoll zu nutzen und immer sauber und ordentlich zu hinterlassen. \emph{Der Schülervertretungs-Raum ist \emph{kein} Lagerraum!} Alles, was nicht innerhalb von zwei Wochen wieder abgeholt wird (Ausnahmen sind nach Absprache mit der Schülervertretung möglich), geht in den Besitz und die Verantwortung der Schülervertretung über.

\Clause{title=Schülervertretung}
Das Schülervertretungs-Team setzt sich aus \emph{den aktuellen Jahrgangssprecher*innen, den Schülersprecher*innen und eventuellen freiwilligen Mitglieder*innen} zusammen.

Die Schülervertretung umfasst \emph{ein Maximum von 16} (in Ausnahmefällen auch 18) Mitglieder*innen, inklusive zwei Schülersprecher*innen. Im Regelfall sollte aus jedem Jahrgang \emph{mindestens ein und maximal drei Mitglieder*innen} vertreten sein. Ausgenommen davon sind die Schülersprecher*innen.

Die Schülervertretung ist in allen schulischen Angelegenheiten, die das Interesse der Schülerschaft berühren, zu beteiligen. Dies schließt die \emph{Vertretung der Schülerschaft in der Gesamtkonferenz, dem Schulvorstand sowie den Fachkonferenzen} ein.

Jedes Schülervertretungs-Mitglied verpflichtet sich zur Mitarbeit, auch in diesen Gremien, und kann ansonsten \emph{jederzeit} nach Ermessen des Schülervertretungs-Teams mit einer $\frac{2}{3}$-Mehrheit ausgeschlossen werden.
 
Die Schülervertretung vertritt, die Schülerschaft und ist offen für alle Ideen, die das Schulleben verbessern könnten. Diese Ideen können jederzeit \emph{mündlich bei unseren Sitzungen, per Zettel in den Schülervertretungs-Briefkasten oder per Mail an \href{mailto:sv-geo@geo-iserv.de}{sv-geo@geo-iserv.de}} eingereicht werden und werden im Normalfall in der nächsten Sitzung besprochen.

Die Schülervertretung tritt \emph{binnen drei Wochen} nach Beginn des neuen Schuljahres in ihrer neu gewählten Form zusammen und hat ab diesem Zeitpunkt \emph{zwei Wochen} Zeit, die Schülersprecher in ihrem Amt zu bestätigen oder bei Bedarf Neuwahlen abzuhalten, sowie über den Verbleib der freiwilligen Mitglieder*innen zu entscheiden. Das fertige Team wird den Klassen- und Kurssprechern auf der nächsten Vollversammlung vorgestellt und bestätigt.

Die Vertreter*innen treffen sich jeden Donnerstag nach der 6. Stunde im Schülervertretungs-Raum. Außerordentliche Treffen werden von den Schülersprecher*innen einberufen. Ebenfalls finden nach Bedarf monatlich Treffen mit der Schulleitung statt. 

Die Mitglieder*innen der Schülervertretung sind den Schüler*innen ihres Jahrgangs zur regelmäßigen Berichterstattung über ihre Tätigkeiten sowie neue Aushänge oder nahende Veranstaltungen verpflichtet.

Die Vertreter verpflichten sich mit dem Eintritt in die Schülervertretung dazu, bei jeder Sitzung anwesend zu sein. Sollte dies nicht möglich sein, muss den Schülersprecher*innen eine Entschuldigung vorliegen oder ein Stellvertreter anwesend sein. Mehrmaliges, unentschuldigtes Fehlen kann zum Ausschluss aus der Schülervertretung führen.

\Clause{title=Schülersprecher*innen}
Es soll \emph{eine Schülersprecherin sowie einen Schülersprecher} geben.

Beide Schülersprecher*innen sind \emph{gleichberechtigt und agieren zusammen} für das Wohl der Schülerschaft. Die Schülersprecher*innen haben den Vorsitz in der Schülervertretung.

Es liegt in der Verantwortung der Schülersprecher*innen, dass durch einen Ausfall (Abitur o.\,ä.) \emph{keine Nachteile für die Schülervertretung} entstehen und ihre/seine Aufgaben (einschließlich Schülervertretungs-Konto) davor zu übergeben.

Schülersprecher*innen werden jährlich von der Schülervertretung gewählt. Grundsätzlich erfolgt die Wahl gegen Ende des Schuljahres (etwa zum Abitur).
Die amtierende Schülervertretung wählt die neuen Schülersprecher*innen mit einfacher Mehrheit. Sollte es eine Stimmengleichheit geben, erfolgt eine Stichwahl. %TODO Widersprüchlich!
Zur Wahl kandidieren kann \emph{jedes Mitglied der Jahrgänge 10-12, welches bereits mindestens ein Jahr aktiv in der Schülervertretung mitgewirkt hat}. Sollte die Schülervertretung mit der Arbeit der Schülersprecher*innen unzufrieden sein, können diese mit einer $\frac{2}{3}$-Mehrheit abgewählt werden. Dann folgen Neuwahlen, bei welchen abgewählte Mitglieder*innen nicht kandidieren dürfen.

Nach gleichem Verfahren wird jährlich aus den Mitglieder*innen der Jahrgänge 5-9 ein(e) Unterstufensprecher*in gewählt. Er/Sie \emph{vertritt explizit die Interessen der Sekundarstufe I} und stellt ein Gegengewicht zu den Schülersprechern dar. Der/Die Unterstufensprecher*in ist \emph{Vertreter*in der Schülersprecher*innen}.

\Clause{title=Jahrgangssprecher*innen}
Die Jahrgangssprecher*innen des 6.-13. Jahrgangs werden \emph{innerhalb der ersten drei Schulwochen} jeden Schuljahres mit einer einfachen Mehrheit von den amtierenden Klassensprechern des jeweiligen Jahrgangs gewählt. Antreten darf jede(r) Schüler*in für ihren/seinen Jahrgang. Jahrgangssprecher*innen sollen die \emph{Interessen ihres Jahrgangs} in der Schülervertretung vertreten und \emph{gleichzeitig die Aufgaben eines Mitglieds erfüllen}. Die Jahrgangssprecher*innen des 5. Jahrgangs werden zu Beginn des 2. Halbjahres gewählt.

\Clause{title=Klassen-/Kurssprecher*innen}
Die Schüler*innen jeder Klasse und jedes Tutorenkurses wählen \emph{bis 2 Wochen nach Schuljahresbeginn} aus ihrer Mitte zwei Klassen-/Kurssprecher*innen.

Die Wahl von Klassen-/Kurssprechern*innen wird von den Klassenlehrern / dem/der Tutor*in oder von ihnen bestimmten Schüler*innen geleitet.
Zur Wahl erfolgt mit \emph{einfacher Mehrheit}. Die Personen, mit der höchsten Stimmenzahl, werden Sprecher*nnen. %TODO: Widersprüchlich!
\emph{Die Wahl wird gültig, sobald die Sieger die Wahl annehmen und in das Klassen-/Kursheft eingetragen werden.}

Die Klassen-/Kurssprecher*innen sind dazu verpflichtet, zu den Vollversammlungen zu erscheinen und müssen dazu bereit sein, die Schülerschaft auf Fachkonferenzen zu vertreten.

Zu den Aufgaben der Klassen-/Kurssprechern*innen gehört, sich aktiv um die Vertretung der Interessen ihrer Klasse bzw. ihres Kurses zu kümmern und intern als Ansprechpartner*in zu dienen. Sie haben das Recht von der Schülervertretung über aktuelle Themen und Entwicklungen der Schule und der Schülervertretung informiert zu werden.

\Clause{title=Weitere Ämter}
Freiwillige Mitglieder:
\begin{itemize}
\item Es können Schüler*innen als freiwillige Mitglieder in das Schülervertretungs-Team aufgenommen werden.
\item Kandidat*innen müssen \emph{eine drei Monate andauernde Probezeit absolvieren}, an deren Ende die Schülervertretung durch $\frac{2}{3}$-Mehrheit über den 
Verbleib im Team entscheidet.
\item Freiwillige Mitglieder*innen sind den Jahrgangssprecher*innen grundsätzlich gleichgestellt, \emph{müssen aber in dem Fall, dass die Mitgliedergrenze überschritten wird damit rechnen, mit einer $\frac{2}{3}$-Mehrheit aus dem Team gewählt zu werden}.
\end{itemize}

Der/Die Protokollant*in
\begin{itemize}
\item hat u.\,a. die Aufgabe, Beschlüsse samt Abstimmungsergebnissen festzuhalten, sowie den ungefähren Verlauf von Schülervertretungs-Sitzungen festzuhalten und eine Anwesenheitsliste zu erstellen.
\end{itemize}

Weitere Ämter
\begin{itemize}
\item Weitere Ämter können nach Bedarf durch Abstimmung der Schülervertretung vergeben werden. Beispiele dafür sind:
\begin{itemize}
\item \emph{Briefkasten und Schülervertretungs-Fach-Beauftragte(r)}
\item \emph{E-Mail-Beauftragte(r)}
\item \emph{Ordnungs- und Inventarbeauftragte(r)}
\item \emph{Finanzbeauftragte(r)}
\end{itemize}
\end{itemize}

\Clause{title=Beschlüsse der Schülervertretung}
Die Schülervertretung beschließt \emph{demokratisch}.

Beschlüsse können \emph{aus eigenem Antrieb oder auf Antrag}, welcher der Schülervertretung zu Beginn der Sitzung vorliegen, getroffen werden.

Antragsberechtigt \emph{sind alle Schüler*innen}. Anträge können nach §~4~Abs.~5 erfolgen. Neben den Schüler*innen können \emph{auch die Schulleitung, Lehrer- oder Elternvertreter*innen Anliegen vortragen}.

Der/Die Antragsteller*in soll an der entsprechenden Sitzung der Schülervertretung teilnehmen, um sein Anliegen vorzutragen.

Für die Annahme eines Antrags wird eine \emph{einfache Mehrheit} benötigt.

\Clause{title=Satzung}
Die Satzung wurde im Einvernehmen der Schülervertretung erstellt und tritt \emph{zu Beginn jedes Schuljahres auf Vorschlag der Schülervertretung mit anschließender Abstimmung der Schülervertretungs-Vollversammlung} durch eine $\frac{2}{3}$-Mehrheit in Kraft.

Zur Änderung dieser Satzung bedarf es eines Änderungsantrages seitens der Schülervertretung, welchem von der Vollversammlung \emph{mit einer $\frac{2}{3}$-Mehrheit} stattgegeben werden muss.

\end{contract}
\vspace{1cm}
Stadt Oldenburg, \today
\end{document}